% \subsection{LLMs for Education}
% Trained on vast repositories of text data, commercial LLMs possess extensive knowledge across a wide range of subjects, allowing them to answer questions, generate content, and support student learning in diverse educational contexts \cite{achiam2023gpt,uppalapati2024comparative}. As a result, educators and researchers have increasingly explored the integration of LLMs into classroom environments.

% Language learning has been a prominent area where LLMs have shown considerable promise. Park et al. \cite{park2024align} and Koraishi et al. \cite{koraishi2023teaching} explored the use of LLM-powered chatbots to help students practice English language skills. These studies demonstrated how conversational agents effectively reinforce grammar, vocabulary, and conversational fluency by simulating real-world dialogue scenarios. Ye et al. \cite{ye2025position} emphasized that LLMs’ multilingual capabilities make them particularly effective tutors for Foreign Language Education (FLE). By adapting language complexity, providing translation assistance, and offering culturally relevant examples, LLMs create interactive learning environments that are tailored to individual learners. Furthermore, Zhang et al. \cite{zhang2024impact} explored how sustained chatbot interactions improved students' vocabulary acquisition and strengthened their command over linguistic structures. 

% By offering contextualized explanations and interactive coding guidance, LLMs have proven to be effective tools in enhancing computational thinking. Ma et al. \cite{ma2024teach} and Jin et al. \cite{jin2024teach} describe how LLMs assist students in debugging code, explaining code logic, and offering step-by-step guidance. Similarly, Tu et al. \cite{tu2023should} explored the integration of LLMs into data science instruction, where the models facilitated data analysis and generated synthetic datasets. Molina et al. \cite{molina2024leveraging} integrated LLM-based tutoring systems in a programming courses and found that LLMs effectively supported learners by clarifying coding terminology and improving comprehension of complex programming concepts. 

% LLMs have also been employed to support creativity, critical thinking, and the humanities. Guo et al. \cite{guo2024prompting} demonstrated LLMs help students explore artistic concepts by generating visual descriptions, creative ideas for design projects. Vastakas et al. \cite{vastakas2024cultural} and Trichopoulos et al. \cite{trichopoulos2023large} demonstrated LLMs effectively supported history education by generating timelines, simulating historical debates, and encouraging inquiry-based exploration. Chang et al. \cite{chang2025framework} demonstrated LLMs can assist students in brainstorming project ideas, generating design concepts, and expanding creative possibilities. Similarly, Zha et al. \cite{zha2024designing} showed that LLMs support project-based learning by co-creating ideas with students, offering design alternatives, and guiding students through iterative ideation processes. By engaging students in interactive dialogue, these tools encouraged reflection, deeper analysis, and creative problem-solving across diverse disciplines.

% The ability of LLMs to generate personalized feedback enhances their value in educational settings. Studies such as those by Dai et al. \cite{dai2023can} and Jia et al. \cite{jia2022automated} demonstrate that LLM-generated feedback on essays, coding exercises, and short-answer questions closely mirrors the quality of feedback provided by human instructors. This immediate, specific feedback allows students to refine their work in real-time, fostering deeper learning. Similarly, Tanwar et al. \cite{tanwar2024opinebot} and Sessler et al. \cite{sessler2025towards} explored LLM-based systems that provide structured critique and targeted suggestions, improving students' ability to reflect, revise, and refine their ideas.




% \subsection{Drawbacks of LLM-Based Education}
